%Arquivo contendo os Resumos
\begin{abstract}
O presente trabalho apresenta o desenvolvimento de um aplicativo "gamificado"\ com a pretensão de motivar seus usuários a praticarem a atividade de ciclismo. A popularização crescente dos \textit{smartphones} e \textit{tablets} trouxe consigo os aplicativos voltados para a saúde pessoal e condicionamento físico. Eles pretendem servir como ferramentas de auxílio ao usuário na prática de uma atividade física, em um cenário em que possuir uma rotina sedentária é comum e a prática constante de exercícios perde espaço. Um aplicativo que usa técnicas de gamificação tem a aspiração de motivar com mais eficiência um usuário que deseje persistir praticando uma atividade física, pois essas técnicas se apropriam dos elementos-chave que os jogos empregam para envolver o ser humano: a socialização, a competição, a fuga da realidade e o aprendizado. Ao longo deste trabalho são descritos os conceitos envolvidos para elaboração do projeto; trabalhos relacionados na mesma temática, os métodos para pesquisa, análise, projeto e desenvolvimento do \textit{software}; as tecnologias implementadas voltadas para Web, incluindo os \textit{frameworks} AngularJS e Bootstrap; o desenvolvimento para \textit{mobile} usando Cordova; o desenvolvimento do \textit{web service} REST com Slim; o uso de geolocalização para monitorar pedaladas do usuário; o módulo de avatar virtual; o módulo de desafios e recompensas e os \textit{softwares} terceiros agregados. Concluindo, são apresentados os resultados obtidos após a implementação da solução técnica, contendo \textit{print screens} das funcionalidades do \textit{software}, os testes envolvendo sessões reais de ciclismo e os objetivos cumpridos. 
\end{abstract}

\begin{englishabstract}
{Pedal-to-Play: A Gamified Web Application to Track Bicycle Rides}
{Health and Fitness application. Geolocation. Pervasive Game. Mobile}

This paper presents the development of a gamified application intended to motivate its users to practice cycling activity. The growing popularity of smartphones and tablets brought applications focused on health and  fitness. They plan to serve as tools in order to aid the user to practice a physical activity, in a scenario that maintaining a sedentary routine is common and practicing exercises loses ground. An application that implements gamification techniques aspires to motivate more efficiently an user that wishes to persist practicing a physical activity, because these techniques take ownership of the key elements that games employ to engage the human being: socializing, competition, escape from reality and learning. Throughout this paper are described the concepts involved in this project development; works related to the same theme, the methods of research, analysis, project design and development of software; the implemented technologies for Web, including the frameworks AngularJS and Bootstrap; development for mobile devices using Cordova; REST web service development with Slim; the use of geolocation to track user rides; the virtual avatar module; the challenges and rewards module and third party softwares aggregated. After the implementation of the technical solution, the results obtained are presented, containing print screens of software features, tests involving real rides and achieved goals.
\end{englishabstract}