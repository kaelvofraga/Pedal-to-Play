%Arquivo contendo o capítulo Conclusões
\chapter{Conclusão} \label{cap:conclusao}
Ao decorrer deste trabalho, dissertou-se sobre as etapas de desenvolvimento do projeto Pedal-to-Play. Inicia apresentando o cenário atual, no qual há uma ascensão do uso de dispositivos \textit{mobile} por toda a população e, em contrapartida, uma queda na prática de atividades físicas. Com a pretensão de unir estes dois mundos, o Pedal-to-Play tem o objetivo de proporcionar um ambiente motivador para a prática de atividades físicas com foco no ciclismo, por meio de um \textit{software} multiplataforma. O principal diferencial deste projeto em relação às outras aplicações existentes com o mesmo objetivo é a utilização de conceitos de gamificação (uso de elementos comuns em jogos para estimular a motivação do usuário). 
\par
Após definir os objetivos e apresentar a proposta do projeto, foram definidos os principais conceitos que permeiam o seu desenvolvimento. O conceito de jogo pervasivo, no qual uma parte da interação com o usuário transcorre no universo físico; de gamificação, explicado como a aplicação de mecanismos de jogos em contextos fora de jogos; de desenvolvimento multiplataforma, consistindo na utilização das tecnologias Web para a solução técnica do projeto; de avatares virtuais, o qual é um elemento comum em jogos digitais; de aplicativos de saúde e \textit{fitness}, referindo-se a categoria de sistemas com o objetivo de integrar tecnologia e atividades físicas e o conceito de geolocalização, consistindo nas técnicas implementadas para monitoramento da atividade do usuário.
\par
Também foram apresentados trabalhos na mesma temática, incluindo as aplicações comerciais Strava e Runtastic Road Bike, ambos possuindo a funcionalidade de monitoramento da atividade de ciclismo. Assim como trabalhos acadêmicos, como o de Cesani e Dranka, propondo um sistema de apoio aos ciclistas de Curitiba e o de Rosero, que desenvolveu soluções em \textit{hardware} e \textit{software} para apoio ao treinamento de ciclistas.
\par 
Contextualizado o sistema, a etapa seguinte envolveu a análise e o projeto da solução técnica. Consistiu na definição da arquitetura do sistema, seguindo o modelo Cliente-Servidor; no levantamento de requisitos, tanto funcionais, quanto não funcionais; na diagramação de casos de uso, definidos a partir dos requisitos funcionais; na prototipação das GUI e na diagramação do modelo de entidade e relacionamento. 
\par
A partir da análise e do projeto, iniciou-se a implementação da solução técnica. A implementação foi dividida em várias entregas e atividades, envolvendo o desenvolvimento dos módulos de Autenticação, de Monitoramento e registro de pedaladas, do Avatar virtual, de Desafios e as atividades de Compilação do projeto para plataforma Android e Hospedagem do \textit{web service} do Pedal-to-Play.
\par
Por fim, foram analisados os resultados obtidos a partir do desenvolvimento da solução técnica e testes monitorando pedaladas reais. Pôde-se afirmar que o Pedal-to-Play conseguiu cumprir grande parte de seus objetivos. A implementação de tecnologias Web permitiu suporte para vários dispositivos, o que conclui o objetivo de ser multiplataforma e de possuir uma interface responsiva, adaptável para diferentes tamanhos de tela. A API de geolocalização do \textit{framework} Cordova possibilitou o monitoramento das pedaladas do usuário e, consequentemente, a análise de dados como distância, tempo e velocidade, essenciais para determinar o desempenho do usuário e fornecer \textit{feedback}, completando o objetivo de analisar pedaladas. Os desafios, o avatar, o mecanismo de recompensas e níveis acrescentaram ao projeto a essência dos jogos e completaram o objetivo do sistema ser gamificado, promovendo a fuga da realidade e competição, os quais fazem parte dos principais fatores motivacionais do ser humano. A integração de todas estas funcionalidades e tecnologias implementadas no Pedal-to-Play tornaram-o um ambiente motivador e de entretenimento, tal qual ele pretendia ser.
\par
Como trabalhos futuros, as seguintes funcionalidades são interessantes para o sistema. A integração com redes sociais (a qual o projeto não conseguiu abordar) é interessante tanto para o usuário compartilhar suas realizações na aplicação (promovendo integração social e competição), quanto para o Pedal-to-Play promover-se nas redes sociais, atraindo mais usuários. A utilização da API do acelerômetro do Cordova, pois possibilitaria um controle mais refinado durante o monitoramento das pedalas, considerando que a geolocalização possui uma margem de erro, o acelerômetro pode ser usado como um complemento para determinar se o usuário está em movimento ou não ou até mesmo se ele está dentro de um veículo e não sobre uma bicicleta. A análise de viabilidade para integração com outras APIs de dados cartográficos, como o OpenStreetMap, pois a API do Google Maps consome uma quantidade significativa de dados de Internet e processamento, o que não é interessante para dispositivos \textit{mobile}. A disponibilidade de desafios compartilhados entre os usuários, como no Strava existem os segmentos, os quais consistem em trajetos no mapa disputados pelos usuários para determinar quem conseguirá os melhores resultados.
\par
E por fim, na parte "gamificada" do sistema, poderão ser desenvolvidos mais níveis, mais desafios (envolvendo diferentes medidas e objetivos), mais recompensas e mais peças para o avatar. Esta possibilidade é totalmente viável, levando em conta que o modo de implementação dos módulos permite dinamismo para acrescentar facilmente mais conteúdo ao sistema.