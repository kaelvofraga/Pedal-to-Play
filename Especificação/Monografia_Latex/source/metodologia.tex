%Arquivo contendo o capítulo Metodologia
\section{Metodologia} \label{sec:metodologia}
As etapas para o desenvolvimento do Pedal-to-Play foram planejadas na seguinte ordem cronológica: levantamento de requisitos; definição dos \textit{frameworks} necessários; definição dos conceitos de gamificação a serem implementados; definição das regras de negócio do sistema e dos casos de uso; análise e modelagem do sistema em UML (Unified Modeling Language\footnote{\url{http://www.uml.org/}}); definição das plataformas suportadas para execução da aplicação; prototipação da interface gráfica com o usuário (GUI\footnote{Acrônimo do inglês para \textit{Graphical User Interface}.}); desenvolvimento das funcionalidades do Pedal-to-Play e concomitantemente a dissertação da monografia e estudo das técnicas necessárias para implementação de cada funcionalidade. Por fim, a etapa de conclusão deste trabalho compreendeu os testes finais das funcionalidades desenvolvidas e a finalização da monografia. 
\par
Durante o decorrer do projeto foram estudados livros, trabalhos relacionados e tutoriais de uso das tecnologias necessárias para o  desenvolvimento do Pedal-to-Play. O material bibliográfico foi pesquisado através de ferramentas de pesquisa eletrônica, como o Google Scholar\footnote{\url{https://scholar.google.com.br/}} e repositórios digitais, como o portal Lume\footnote{\url{http://www.lume.ufrgs.br/}}, da Universidade Federal do Rio Grande do Sul. As palavras chave que direcionaram a pesquisa foram: \textit{mobile}, ciclismo, \textit{gamification}, \textit{web}, \textit{rastreamento de atividades} e geolocalização. Os tutoriais foram buscados em plataformas de ensino digital, como o Mozilla Developer Network (MDN) e nos \textit{sites} oficiais de cada tecnologia. \par
Os produtos alvo foram o \textit{software} desenvolvido (gratuito e de código aberto) e a monografia, descrevendo o processo de pesquisa e desenvolvimento.
