%Arquivo contendo o capítulo Introdução
\chapter{Introdução} \label{cap:introducao}
Atualmente, \textit{smarthphones} e aplicativos para acompanhamento de atividades físicas e saúde pessoal estão bem popularizados, eles buscam motivar seus usuários a continuar praticando exercícios e dedicando tempo de suas vidas para saúde. Todavia, esses resultados não são instantâneos e nesta época de bombardeio de informações, o \textit{feedback} demorado da prática de um exercício vem a desmotivar os iniciantes, entre outros fatores, como tais: a exaustiva rotina diária e a falta de acompanhamento para a atividade. \par 

Nestes pontos fracos que os aplicativos modernos de acompanhamento de atividades físicas atacam, se tornando companheiros de bolso para qualquer um que deseje praticar um esporte, mesmo que por lazer. Eles proporcionam a socialização através das redes sociais, \textit{feedback} instantâneo através dos dados coletados via funcionalidades dos \textit{smarthphones} e competição por melhores desempenhos entre os usuários do mesmo aplicativo. \par

O modo de vida contemporâneo é outro fator que não favorece a prática de exercícios físicos, pois a maior parte do dia da população é ocupada por trabalho ou estudo, senão ambos. Entre estes dois também há o quesito trânsito. O tempo livre restante é aproveitado para descanso ou socialização com família ou com amigos. Aqueles que praticam esportes por \textit{hobbie} encontram, ao passar o tempo, cada vez mais obstáculos para praticá-los: seja por falta de tempo, por problemas econômicos e financeiros ou por simplesmente não sentir prazer na atividade praticada \cite{butcher2002, liz2013}. \par

Vivendo neste cenário, onde a prática de uma atividade física não se torna hábito em nossas vidas (embora seja necessário para manter nosso corpo e mente saudáveis) surge o questionamento, de que maneira um \textit{software} poderia motivar os usuários a persistirem na prática de um exercício, mais especificamente, a atividade de ciclismo? \par

Assim, a pesquisa fruto desse projeto pretende desenvolver um \textit{software} multiplataforma (\textit{mobile} e \textit{desktop}) que use as tecnologias de desenvolvimento \textit{Web} e aplique conceitos de gamificação no formato de um jogo pervasivo, o qual será denominado Pedal-to-Play. \par

Entre as funcionalidades do Pedal-to-Play, propõe-se a captura e processamento de determinados dados durante as seções de ciclismo do usuário, com o fim de mensurar o desempenho dele na atividade, recompensando-o com pontos e \textit{badges} (troféus virtuais e indicadores de realização de tarefas) de acordo com os resultados alcançados. Pretende-se possibilitar o compartilhamento das recompensas em redes sociais. O \textit{software} também possuirá um sistema de \textit{avatar} virtual customizável, o qual ilustrará o perfil do usuário, contendo informações sobre a evolução do mesmo na atividade de ciclismo.

\section{Motivação}
Diferente dos \textit{softwares} que somente coletam os dados do usuário e os exibem de forma gráfica, um sistema que implementa técnicas de gamificação em seu desenvolvimento propõe alternativas para a superação de fatores desmotivadores na prática de uma atividade. As técnicas de gamificação  trabalham estimulando os principais fatores motivadores para o ser humano: competição, aprendizado, fuga da realidade e interação social \cite{vianna2013}. Assim, o Pedal-to-Play proporá desafios ao usuário e o mesmo será recompensado pela participação neles. Buscando concomitantemente conciliar o entretenimento na atividade exercida pelo usuário e permitindo a socialização com outros ciclistas que também façam uso do \textit{software}. \par

A atividade de ciclismo foi escolhida como foco do trabalho pelos benefícios a saúde do praticante, ao meio ambiente e a mobilidade urbana. \citet{rojasrueda2011} constatou que o aumento de adeptos à bicicleta como meio de transporte na cidade de Barcelona ocasionou na diminuição de acidentes no trânsito e a diminuição de gás carbônico no ambiente.

\section{Objetivo}
Este trabalho possui os seguintes objetivos:

\subsection{Objetivo Geral}
Desenvolver um \textit{software} com a pretensão de motivar o usuário a persistir praticando a atividade de ciclismo. 

\subsection{Objetivos Específicos}
\begin{itemize}
\item Identificar quais dados devem ser analisados e como computá-los para mensurar o desempenho do ciclista;
\item Identificar e selecionar quais \textit{frameworks} e linguagens de programação adequados para o desenvolvimento do \textit{software};
\item Desenvolver uma interface homem-computador responsiva e multiplataforma;
\item Desenvolver um mecanismo dentro da aplicação para propor desafios ao usuário e recompensá-lo pela participação nestes;
\item Permitir ao usuário compartilhar seus dados contidos no \textit{software} em redes sociais;
\item Analisar e aplicar técnicas de gamificação para o desenvolvimento do \textit{software}.
\end{itemize}

\section{Organização do Texto}
Este documento está divido em três capítulos. O capítulo \ref{cap:metodologia} versa os métodos que serão utilizados para a elaboração da pesquisa e o desenvolvimento do \textit{software}, assim como descreve as etapas do trabalho e as funcionalidades que pretende-se implementar. O capítulo \ref{cap:cronograma} conterá uma tabela contendo as tarefas fundamentais deste trabalho e em qual mês cada uma delas será efetuada.
O cronograma abrange quatro meses do segundo semestre de 2015.

