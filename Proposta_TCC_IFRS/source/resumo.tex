%Arquivo contendo os Resumos
\begin{abstract}
O presente trabalho propõe o desenvolvimento de um aplicativo "gamificado" com a pretensão de motivar seus usuários a praticarem a atividade de ciclismo. A popularização crescente dos \textit{smartphones} e \textit{tablets} trouxe consigo os aplicativos voltados para a saúde pessoal e condicionamento físico. Eles pretendem servir como ferramentas de auxílio ao usuário na prática de uma atividade física, em um cenário em que possuir uma rotina sedentária é comum e a prática constante de exercícios perde espaço. Um aplicativo que usa técnicas de gamificação tem a aspiração de motivar com mais eficiência um usuário que deseje persistir praticando uma atividade física, pois essas técnicas se apropriam dos elementos chave que os jogos empregam para envolver o ser humano: a socialização, a competição, a fuga da realidade e o aprendizado. Ao longo desta proposta serão descritos os conceitos envolvidos para elaboração do trabalho, as tecnologias que pretende-se utilizar, os métodos para pesquisa e desenvolvimento do \textit{software} e alguns trabalhos relacionados já elaborados.
\end{abstract}

% resumo na outra lingua (opcional)
%\begin{englishabstract}
%{Title of the Work in English}
%{System. ABNT. IFRS} % Palavras Chaves: iniciar com letras maiúsculas e %separar por '.'
%This work has the purpose of [...]. The text in the abstract should not %contain more than 500 words.
%\end{englishabstract}